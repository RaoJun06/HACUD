\section{Conclusion}
%In this paper, we firstly study the cash-out user detection problem from network structure perspective and propose a novel HACUD model for the purpose. To characterize the complex and heterogeneous cash-out fraud scene, we model the plentiful attributes of users and rich relations between users as an attributed heterogeneous information network. Next, we design a hierarchical attention mechanism to model user's preferences towards attribute information and meta-paths. Finally, with the real datasets in Ant Credit Pay of Ant Financial Services Group, we do extensive experiments for the cash-user detection problem, and demonstrate the effectiveness of the proposed model compared to the state of arts. As future work, we will investigate to integrating more heterogeneous information (\ie more relative objects and interaction relations between them) and extending our model to semi-supervised scene.

In this paper, we first study the cash-out user detection problem under the attributed heterogenous information network
framework, constituted by objects and their relations in the scenario of credit payment service, and propose a novel HACUD model
for the purpose. With the help of meta-path based neighbors, we aggregate features of objects  from node attributes,
as well as structure features generated by meta-paths. Furthermore, we design a hierarchical attention mechanism to
model user preferences towards attributes and meta-paths. With the real datasets in Ant Credit Pay of Ant Financial Services Group, extensive experiments for the cash-user detection  task demonstrate the effectiveness of our model. As future work, we will investigate to integrating more heterogeneous information (\eg interaction relations) and extending our model to semi-supervised scenario.

%As future work, we will investigate to 
%integrating more heterogeneous information and 
%extending our model to semi-supervised scene.